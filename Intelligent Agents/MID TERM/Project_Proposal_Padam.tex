\documentclass{article}

\title{Adaptive Motion Planning with Artificial Potential Fields (APF) for UGVs: Dynamic and Static Target Tracking with Obstacle Avoidance}
\author{Name: Padam Jung Thapa (ID: 2623560)}
\date{Date of Proposal: 29th February 2024}
\usepackage[margin=1in]{geometry} 
\usepackage{graphicx}


\begin{document}

\maketitle

\begin{center}
\section*{MID-TERM PROPOSAL}
\section*{CSCI 6550: Intelligent Agents}

Instructor Name: Dr. Abdullah Al Redwan Newaz \\
University of New Orleans
\end{center}

\section*{Motivation}
The field of robotics has seen significant advancements in autonomous navigation, particularly in the development of obstacle avoidance strategies for mobile robots. A critical aspect of autonomous navigation is the ability of a robot to reach a designated target while avoiding collisions with obstacles. This proposal outlines the implementation of an Artificial Potential Field (APF) method for obstacle avoidance using the mobile robot in CoppeliaSim, focusing on the pursuit of a random static target as well as the dynamic target which exhibits erratic movement patterns.

\section*{Objective}
The primary objective of this study is to enhance the APF algorithm to enable the Pioneer P3dX mobile robot to trace a path and avoid obstacles effectively along with tracking a particular human amidst obstacles. The APF method is widely used due to its simplicity and real-time application capabilities. The project will showcase the robot's ability to adapt its trajectory in real-time to follow the moving target while maintaining a safe distance from obstacles by utilizing the sensor data.

\section*{Related Works}
Previous studies have explored various obstacle avoidance strategies, ranging from geometric methods to intelligent algorithms. The Bug algorithms, such as Bug1 and Bug2, are simple and effective but can result in longer paths and are not suitable for dynamic environments. Vector Field Histogram (VFH) and Dynamic Window Approach (DWA) are more advanced, considering the robot's dynamics and sensor data to navigate around obstacles. The APF method has been criticized for issues like local minima, where a robot might get stuck in an area surrounded by obstacles [1]. Recent advancements have proposed solutions to mitigate these issues, such as hybridizing APF with other techniques or modifying the potential field functions. The APF technique has also been used in the path planning for the obstacle avoidance in a 3D Environment by an Unmanned Aerial Vehicle (UAVs) [5].

\section*{Artificial Potential Field}
An artificial potential field (APF) is a mathematical construct used in robotics and path planning to guide a robot's motion [6]. It is a scalar field that assigns a value to each point in the environment, representing the potential energy of a robot at that point. The robot's motion is then guided by the gradient of the potential field, which points in the direction of the lowest potential energy.

\section*{Attractive/Repulsive Potential Field}
In the APF approach to robot navigation, the movement of the robot is influenced by two types of virtual forces: an attractive force that pulls the robot towards the goal, and a repulsive force that pushes the robot away from obstacles. The attractive force is a function of the robot's distance and bearing to the target, while the repulsive force depends on the robot's proximity and orientation relative to nearby obstacles. In the depicted force diagram, \( F_1 \) represents the attractive force directed towards the target, \( F_2 \) symbolizes the repulsive force exerted by an obstacle, and the vector \( \mathbf{F} \) is the net/resultant force that dictates the robot's motion, resulting from the combination of \( F_1 \) and \( F_2 \) [1][2]. The attractive function has a single global minimum in the goal configuration, however when combined with the repulsive function local minima might appear in some of the complex configurations, which is further addressed by the dynamic window approach in some studies [3].


\begin{figure}[h!]
  \centering
  \begin{minipage}{0.48\textwidth}
    \centering
    \includegraphics[width=\linewidth]{a.png} 
    \caption{Definition of attractive force and repulsive force in an artificial potential field.}
    \label{fig:image1}
  \end{minipage}\hfill 
  \begin{minipage}{0.48\textwidth}
    \centering
    \includegraphics[width=\linewidth]{b.png} 
    \caption{Static Representation of an APF with its gradient field.}
    \label{fig:image2}
  \end{minipage}
\end{figure}

The image provided is a static representation of an APF scenario, where we can see a gradient field that likely represents the potential field generated by the attractive and repulsive forces. The darker areas indicate higher repulsive potentials from obstacles, while the lighter areas suggest lower potential, making them more favorable paths for the robot to follow [4].



\begin{thebibliography}{9}
\bibitem{ref1} Sung, Nakyeong, Suhwan Kim, and Namsuk Cho. 2023. "An Efficient Path Planning Algorithm Using a Potential Field for Ground Forces."
\bibitem{ref2} Zhang et. al. "Artificial potential field based anti-saturation positioning obstacle avoidance control for wheeled robots."
\bibitem{ref3} \url{https://www.cs.cmu.edu/~motionplanning/lecture/Chap4-Potential-Field_howie.pdf}

\bibitem{ref4} \url{https://github.com/AtsushiSakai/PythonRobotics/}
\bibitem{ref5} Ana Batinovic, Jurica Goricanec, Lovro Markovic, Stjepan Bogdan. 2023 "Path Planning with Potential Field-Based Obstacle Avoidance in a 3D Environment by an Unmanned Aerial Vehicle."
\bibitem{ref6} Iswanto, Alfian Ma'arif. 2019 "Artificial Potential Field Algorithm Implementation for Quadrotor Path Planning."
\end{thebibliography}


\end{document}